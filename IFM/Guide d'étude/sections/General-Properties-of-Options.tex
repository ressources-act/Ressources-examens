\chapter[General Properties of Options]{General Properties of Options (10\% à 15\%)}

\subsection{Information}

\begin{distributions}[Objective]
The Candidate will understand how call options and put options can be used in conjunction with the underlying asset in a risk management context.
\end{distributions}

\begin{outcomes}[Learning outcomes]
The Candidate will be able to:
\begin{enumerate}[label = \alph*)]
	\item	Explain the cash flow characteristics and terms relating to various options.
		\begin{knowledge}[]
		\begin{itemize}
		\item	Define and recognize the following terms relating to option classification: 
			\begin{itemize}
			\item	call and put options, 
			\item	expiration date, 
			\item	strike price,
			\item	moneyness, 
			\item	and option style.
			\end{itemize}
		\item	Calculate the payoff and profit on both long and short positions with respect to both call and put options.
		\item	Calculate the payoffs on exotic options: 
			\begin{itemize}
			\item	Asian (arithmetic and geometric), 
			\item	barrier, 
			\item	compound, 
			\item	gap, and 
			\item	exchange.
			\end{itemize}
		\item	Calculate the payoffs on exotic options: 
			\begin{itemize}
			\item	lookback, 
			\item	chooser, 
			\item	shout, 
			\item	rainbow, and 
			\item	forward start.
			\end{itemize}
		\end{itemize}
		\end{knowledge}
	\item	Apply option strategies in a risk management context.
		\begin{knowledge}[]
		\begin{itemize}
		\item	Recognize that a long put can be used as an insurance strategy for a long stock position and a long call can be used as an insurance strategy for a short stock position.
		\item	Understand how the following option strategies can be used as tools to manage financial risk or speculate on price or volatility: 
			\begin{itemize}
			\item	option spreads (bull, bear, ratio), 
			\item	collar, 
			\item	straddle, 
			\item	strangle, and 
			\item	butterfly spread.
			\end{itemize}
		\item	Evaluate the payoff and profit of the option strategies described above.
		\end{itemize}
		\end{knowledge}
	\item	Explain the general properties of options that affect option prices.
		\begin{knowledge}[]
		\begin{itemize}
		\item	Apply put-call parity to European options on stocks with
			\begin{itemize}
			\item	no dividends, 
			\item	continuous dividends, 
			\item	discrete dividends,
			\item	currencies, and 
			\item	bonds.
			\end{itemize}			 
		\item	Compare options with respect to term-to-maturity and strike price.
		\item	Identify factors affecting the early exercise of American options and the situations where the values of European and American options are the same.
		\end{itemize}
		\end{knowledge}
	\end{enumerate}
\end{outcomes}

\begin{ASM_chapter}[Related lessons ASM]
\begin{enumerate}
  \setcounter{enumi}{15}
	\item	\nameref{L.-16}
	\item	\nameref{L.-17}
	\item	\nameref{L.-18}
\end{enumerate}
\end{ASM_chapter}

\begin{YTB_vids}[Vidéos YouTube]
\begin{itemize}
	\item	
\end{itemize}
\end{YTB_vids}

\subsection{Résumés des chapitres}

\begin{CHPT_SUMM_AUTO}[label = {L.-16}]{16. Options}
	\begin{itemize}
		\item	
	\end{itemize}
\end{CHPT_SUMM_AUTO}

\begin{CHPT_SUMM_AUTO}[label = {L.-17}]{17. Option Strategies}
	\begin{itemize}
		\item	
	\end{itemize}
\end{CHPT_SUMM_AUTO}

\begin{CHPT_SUMM_AUTO}[label = {L.-18}]{18. Put-Call Parity}
	\begin{itemize}
		\item	
	\end{itemize}
\end{CHPT_SUMM_AUTO}

\subsection{Notes sur les vidéos YouTube}

%\begin{YTB_SUMM}[label = {}]{}
%\begin{itemize}
%	\item	
%\end{itemize}
%\end{YTB_SUMM}