\chapter[Introductory Derivatives---Forwards and Futures]{Introductory Derivatives---Forwards and Futures (5\% à 10\%)}

\subsection{Information}

\begin{distributions}[Objective]

\end{distributions}

\begin{outcomes}[Learning outcomes]
\begin{enumerate}
	\item	
\end{enumerate}
\end{outcomes}

\begin{ASM_chapter}[Related lessons ASM]
\begin{itemize}
	\item	\nameref{L.-1}
	\item	\nameref{L.-14}
	\item	\nameref{L.-15}
\end{itemize}
\end{ASM_chapter}

\begin{YTB_vids}[Vidéos YouTube]
\begin{itemize}
	\item	
\end{itemize}
\end{YTB_vids}

\subsection{Résumés des chapitres}

\begin{CHPT_SUMM_AUTO}[label = {L.-1}]{1. Introduction to Derivatives}
	\begin{itemize}
	\item	What is a derivative
		\begin{itemize}
		\item	\textbf{Derivative}: Financial \og instrument \fg{} whole value is determined by the price of something else.
		\item	For example, if I'm a farmer and I strike a deal such that if the price of corn drops below 3\$, a client will pay me 1\$ but that if it goes above 3\$ I will pay the client 1\$;
		\item[]	This way, I'm insured against the risk of my corn dropping and he's insured against the risk of the price shooting up---the risk is reduced for the both of us.
		\item	If I were just an investor and not the farmer himself, I could speculate on what the price will be in which case the derivative would be a \textbf{bet} and not an insurance;
		\item	As such \textit{the use} of the contrat, and not the contract itself, determines whether it's \textbf{risk-reducing};
		\end{itemize}
	\item	An overview of financial markets
		\begin{itemize}
		\item	Trading of financial assets;
			\item	Usually with stock and bond trades, the buyer and seller have no continuing obligations but that is not the case with derivatives;
			\item	So there's a \textit{clearinghouse} to manage the transaction;
		\item	For large sellers and buyers, they can do \textit{over-the-counter (OTC)} trading;
		
		\end{itemize}
	\item	The role of financial markets
	\item	The use of derivatives
	\item	Buying and short-selling financial assets
	\end{itemize}
\end{CHPT_SUMM_AUTO}

\begin{CHPT_SUMM_AUTO}[label = {L.-14}]{14. Forwards}
	\begin{itemize}
		\item	
	\end{itemize}
\end{CHPT_SUMM_AUTO}

\begin{CHPT_SUMM_AUTO}[label = {L.-15}]{15. Variations on the Forward Concept}
	\begin{itemize}
		\item	
	\end{itemize}
\end{CHPT_SUMM_AUTO}

\subsection{Notes sur les vidéos YouTube}

%\begin{YTB_SUMM}[label = {}]{}
%\begin{itemize}
%	\item	
%\end{itemize}
%\end{YTB_SUMM}
