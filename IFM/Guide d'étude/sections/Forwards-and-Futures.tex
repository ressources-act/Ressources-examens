\chapter[Introductory Derivatives---Forwards and Futures]{Introductory Derivatives---Forwards and Futures (5\% à 10\%)}

\subsection{Information}

\begin{distributions}[Objective]
The Candidate will understand how forward contracts and futures contracts can be used in conjunction with the underlying asset in a risk management context.
\end{distributions}

\begin{outcomes}[Learning outcomes]
The Candidate will be able to:
\begin{enumerate}[label = \alph*)]
	\item	Describe the characteristics and terms of the main derivatives instruments (including forwards and futures).
		\begin{knowledge}[]
		\begin{itemize}
		\item	Distinguish between long and short positions for both assets (including short selling of stocks) and derivatives on assets.
		\item	Recognize the transaction costs affecting profit calculations for both assets and derivatives on assets (including commissions and bid-ask spread).
		\end{itemize}
		\end{knowledge}
	\item	Describe the characteristics and terms relating to both forward contracts and prepaid forward contracts.
		\begin{knowledge}[]
		\begin{itemize}
		\item	Define and recognize the following terms relating to the timing of stock purchases: outright purchase, fully leveraged purchase, prepaid forward contract, and forward contract.
		\item	Determine payoffs and profits for both long and short positions on forward contracts.
		\item	Calculate prices for both forward contracts and prepaid forward contracts on stocks with no dividends, continuous dividends, and discrete dividends. 
		\item	Construct a synthetic forward from the underlying stock and a risk-free asset and identify arbitrage opportunities when the synthetic forward price is different from the market forward price.
		\end{itemize}
		\end{knowledge}
	\item	Describe the characteristics and terms relating to both futures contracts and the associated margin accounts.
		\begin{knowledge}[]
		\begin{itemize}
		\item	Define and recognize the following terms relating to the mark-to-market process: Marking to market, margin balance, maintenance margin, and margin call. 
		\item	Evaluate an investor’s margin balance based on changes in asset values.
		\end{itemize}
		\end{knowledge}
	\end{enumerate}
\end{outcomes}

\begin{ASM_chapter}[Related lessons ASM]
\begin{itemize}
	\item	\nameref{L.-1}
	\item	\nameref{L.-14}
	\item	\nameref{L.-15}
\end{itemize}
\end{ASM_chapter}

\begin{YTB_vids}[Vidéos YouTube]
\begin{itemize}
	\item	
\end{itemize}
\end{YTB_vids}

\subsection{Résumés des chapitres}

\begin{CHPT_SUMM_AUTO}[label = {L.-1}]{1. Introduction to Derivatives}
	\begin{itemize}
	\item	What is a derivative
		\begin{itemize}
		\item	\textbf{Derivative}: Financial \og instrument \fg{} whole value is determined by the price of something else.
		\item	For example, if I'm a farmer and I strike a deal such that if the price of corn drops below 3\$, a client will pay me 1\$ but that if it goes above 3\$ I will pay the client 1\$;
		\item[]	This way, I'm insured against the risk of my corn dropping and he's insured against the risk of the price shooting up---the risk is reduced for the both of us.
		\item	If I were just an investor and not the farmer himself, I could speculate on what the price will be in which case the derivative would be a \textbf{bet} and not an insurance;
		\item	As such \textit{the use} of the contrat, and not the contract itself, determines whether it's \textbf{risk-reducing};
		\end{itemize}
	\item	An overview of financial markets
		\begin{itemize}
		\item	Trading of financial assets;
			\item	Usually with stock and bond trades, the buyer and seller have no continuing obligations but that is not the case with derivatives;
			\item	So there's a \textit{clearinghouse} to manage the transaction;
		\item	For large sellers and buyers, they can do \textit{over-the-counter (OTC)} trading;
		
		\end{itemize}
	\item	The role of financial markets
	\item	The use of derivatives
	\item	Buying and short-selling financial assets
	\end{itemize}
\end{CHPT_SUMM_AUTO}

\begin{CHPT_SUMM_AUTO}[label = {L.-14}]{14. Forwards}
	\begin{itemize}
		\item	
	\end{itemize}
\end{CHPT_SUMM_AUTO}

\begin{CHPT_SUMM_AUTO}[label = {L.-15}]{15. Variations on the Forward Concept}
	\begin{itemize}
		\item	
	\end{itemize}
\end{CHPT_SUMM_AUTO}

\subsection{Notes sur les vidéos YouTube}

%\begin{YTB_SUMM}[label = {}]{}
%\begin{itemize}
%	\item	
%\end{itemize}
%\end{YTB_SUMM}
