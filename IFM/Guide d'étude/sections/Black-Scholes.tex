\chapter[Black-Scholes Option Pricing Model]{Black-Scholes Option Pricing Model (10\% à 15\%)}

\subsection{Information}

\begin{distributions}[Objective]
The Candidate will understand how the Black-Scholes Formula can be used to form the prices of European call and put options on various underlying assets.
\end{distributions}

\begin{outcomes}[Learning outcomes]
The Candidate will be able to:
\begin{enumerate}[label = \alph*)]
	\item	Explain the properties of the lognormal distribution and its applicability to option pricing.
		\begin{knowledge}[]
		\begin{itemize}
		\item	Calculate lognormal-based probabilities and percentiles for stock prices. 
		\item	Calculate lognormal-based means and variances of stock prices.
		\item	Calculate lognormal-based conditional expectations of stock prices given that options expire in-the-money.
		\end{itemize}
		\end{knowledge}
	\item	Explain the Black-Scholes Formula.
		\begin{knowledge}[]
		\begin{itemize}
		\item	Recognize the assumptions underlying the Black-Scholes model.
		\item	Estimate a stock’s historical volatility from past stock price data.
		\item	Use the Black-Scholes Formula to value European calls and puts on stocks with no dividends, stock indices with continuous dividends, stocks with discrete dividends, currencies, and futures contracts.
		\item	Generalize the Black-Scholes Formula to value 
			\begin{itemize}
			\item	gap calls,
			\item	gap puts
			\item	exchange options, 
			\item	chooser options, and 
			\item	forward start options.
			\end{itemize}
		\end{itemize}
		\end{knowledge}
	\end{enumerate}
\end{outcomes}

\begin{ASM_chapter}[Related lessons ASM]
\begin{itemize}
	\item	\nameref{L.-23}
	\item	\nameref{L.-24}
	\item	\nameref{L.-27}
\end{itemize}
\end{ASM_chapter}

\begin{YTB_vids}[Vidéos YouTube]
\begin{itemize}
	\item	
\end{itemize}
\end{YTB_vids}

\subsection{Résumés des chapitres}

\begin{CHPT_SUMM_AUTO}[label = {L.-23}]{23. Modeling Stock Prices with the Lognormal Distribution}
	\begin{itemize}
		\item	
	\end{itemize}
\end{CHPT_SUMM_AUTO}

\begin{CHPT_SUMM_AUTO}[label = {L.-24}]{24. The Black-Scholes Formula}
	\begin{itemize}
		\item	
	\end{itemize}
\end{CHPT_SUMM_AUTO}

\begin{CHPT_SUMM_AUTO}[label = {L.-27}]{{27. Asian, Barrier, and Compound Options}}
	\begin{itemize}
		\item	
	\end{itemize}
\end{CHPT_SUMM_AUTO}

\subsection{Notes sur les vidéos YouTube}

%\begin{YTB_SUMM}[label = {}]{}
%\begin{itemize}
%	\item	
%\end{itemize}
%\end{YTB_SUMM}
