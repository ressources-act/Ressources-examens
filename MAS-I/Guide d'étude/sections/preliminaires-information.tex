\section{Information}

\begin{distributions}[Objectives]
\begin{itemize}
	\item	Set forth, usually in broad terms, what the candidate should be able to do in actual practice;
	\item	The objectives include methodologies that may be impossible to perform on an exam that the candidate is expected to be able to explain conceptually;
	\item	\textit{For example}: The Hat Matrix couldn't be calculated, but conceptual questions about it could be asked;
\end{itemize}
\end{distributions}

\begin{outcomes}[Learning outcomes]
\begin{enumerate}
	\item	It's important to identify some of the key terms, concepts, and methods associated with each of the learning objectives;
	\item	They aren't an exhaustive list of the material being tested, but rather illustrate the scope of each learning objective;
\end{enumerate}
\end{outcomes}

\begin{CHPT_SUMM}{Information additionnelle}
	\begin{itemize}
		\item	The learning objectives define the behaviours and the knowledge statements illustrate more fully their intended scope;
		\item	Learning objectives should not be seen as independent units but as building blocks for our understanding;
		\item	The ranges are just guidelines;
		\item	The overall section weights should be seen as having more significance than the individual section weights;
		\item	Tables include :
			\begin{itemize}
			\item	values for the illustrative life tables;
			\item	Standard normal distribution;
			\item	Abridged inventories of discrete and continuous probability distributions;
			\item	Chi-square distribution;
			\item	$t$-distribution;
			\item	$F$-distribution;
			\end{itemize}
		\item	\textbf{There is a guessing adjustment};
	\end{itemize}
\end{CHPT_SUMM}
