\documentclass[english]{article}
%% -----------------------------
%% Préambule
%% -----------------------------
\input{../cheatsht-preamble-general-en.tex}

%% -----------------------------
%% Variable definition
%% -----------------------------
\def\cours{Modern Actuarial Statistics II}
\def\sigle{MAS-II}

%
% 	Save more space than default
%
\setlength{\abovedisplayskip}{-15pt}
\setlist{leftmargin=*}

%% -----------------------------
%% 	Colour setup for sections
%% -----------------------------
\def\SectionColor{cobalt}
\def\SubSectionColor{azure(colorwheel)}
\def\SubSubSectionColor{azure(colorwheel)}
%%%	depth
\setcounter{secnumdepth}{0}
%% -----------------------------

%% -----------------------------
%% 	Format part
%% -----------------------------
\titleformat
	{\part}			%	command
	[display]		%	shape
	{\normalfont\bfseries\filcenter}	%	format
	{\LARGE\thepart}	%	label
	{1ex}			%	sep
	{
		\titlerule[2pt]
		\vspace{2ex}%
		\LARGE
	}				%	before-code
	[
		\vspace{1ex}%
		{\titlerule[2pt]}
	]				%	after-code
\titlespacing{\part}
	{0pc}			%	left margin spacing
	{-5mm}			%	vertical space before title
	{1pc}			%	seperation between title and non-sectioning text
\renewcommand\thepart{\Alph{part}}
%% -----------------------------

%% -----------------------------
%% Color definitions
%% -----------------------------
\definecolor{indigo(web)}{rgb}{0.29, 0.0, 0.51}
\definecolor{cobalt}{rgb}{0.0, 0.28, 0.67}
\definecolor{azure(colorwheel)}{rgb}{0.0, 0.5, 1.0}
%% -----------------------------

%% -----------------------------
%% Variable definition
%% -----------------------------
%%
%% Matrix notation variable (bold style)
%%
\newcommand\cololine[2]{\colorlet{temp}{.}\color{#1}\bar{\color{temp}#2}\color{temp}}
\newcommand\colbar[2]{\colorlet{temp}{.}\color{#1}\bar{\color{temp}#2}\color{temp}}
\newcommand\cumlaut[2][black]{\stackon[.33ex]{#2}{\textcolor{#1}{\kern-.04ex.\kern-.2ex.}}}

\begin{document}

\begin{center}
	\textsc{\Large Contributeurs}\\[0.5cm] 
\end{center}
\begin{contrib}{MAS-II: Modern Actuarial Statistics II}
\begin{description}
	\item[aut., cre.] Alec James van Rassel
\end{description}

\textbf{\underline{Référence (manuels, YouTube, notes de cours)}}
En ordre alphabétique :
%\begin{description}
%	\item[src.]	Coaching Actuaries, Coaching Actuaries MAS-I Manual.
%	\item[src.]	Côté, M.-P., ACT-2000 : Analyse statistique des risques actuariels, Université Laval, Québec (QC).
%	\item[src.]	Hogg, R.V.; McKean, J.W.; and Craig, A.T., Introduction to Mathematical Statistics, 7th Edition, Prentice Hall, 2013.
%\end{description}

\textbf{\underline{Contributeurs}}
%\begin{description}
%	\item[pfr.]	Sharon van Rassel
%\end{description}
\end{contrib}

%\begin{distributions}[Cours reliés]
%\begin{description}
%	\item[ACT-2008]	Mathématiques actuarielles IARD II
%\end{description}

%En partie : mathématiques actuarielles vie I (\textbf{ACT-2004}), séries chronologiques (\textbf{ACT-2010}), introduction à l'actuariat II (\textbf{ACT-2001}) et méthodes numériques (\textbf{ACT-2002}).
%\end{distributions}


%\begin{rappel_enhanced}[Motivation]
%
%\end{rappel_enhanced}



\newpage
\raggedcolumns
\begin{multicols*}{2}
\tableofcontents



\newpage
\part{Introduction to Credibility}\label{part:cred}
\section{Basic Framework of Credibility}\label{sec:credBasics}
\begin{rappel_enhanced}[Context]
The \textbf{\textit{limitation fluctuation credibility}} approach, or \textbf{\textit{classical credibility}} approach, calculates an updated prediction ($U$) of the \textbf{loss measure} as a weighted ($Z$) average of recent claim experience ($D$) and a rate ($M$) specified in the manual. Thus, we calculate the \textit{premium} paid by the \textit{risk group} as \lfbox[formula]{$U = ZD + (1 - Z)M$}.
\end{rappel_enhanced}

\begin{distributions}[Notation]
\begin{description}
	\item[$M$]	Predicted loss based on the "\textit{\textbf{m}anual}".
	\item[$D$]	Observe\textbf{d} losses based on the recent experience of the risk group.
	\item[$Z$]	Weight assigned to the recent experience $D$ called the \textbf{\textit{credibility factor}} with \lfbox[conditions]{$Z \in [0, 1]$}. 
	\item[$U$]	\textbf{U}pdated prediction of the premium.
\end{description}
\end{distributions}

\begin{distributions}[Terminology]
\begin{description}
	\item[Risk group]	block of insurance policies, covered for a period of time upon payment of a \textit{premium}.
	\item[Claim frequency]	The number of claims denoted $N$.
	\item[Claim severity]	The amount of the $i^{\text{th}}$ claim denoted $X_{i}$.
	\item[Aggregate loss]	The total loss denoted $S$ where $S = X_{1} + X_{2} + \hdots + X_{N}$.
	\item[Pure premium]	The pure premium denoted $P$ where $P = S/E$ with $E$ denoting the number of exposure units.
\end{description}
\end{distributions}

\begin{definitionGENERAL}{Exam tips}[][ao(english)]
Typical questions about this involve being given 3 of $M, D, Z, \text{ and } U$ then finding the missing one.
\end{definitionGENERAL}


\begin{rappel_enhanced}[Context]
With $\min\{D, M\} \leq U \leq \max\{D, M\}$, we can see that the credibility factor determines the relative importance of the claim experience of the risk group $D$ relative to the manual rate $M$.

\bigskip

If $Z = 1$, we obtain \textit{\underline{\nameref{subsec:FullCred}}} where the predicted premium depends only on the data ($U = D$). It follows that with $Z < 1$, we obtain \textit{\underline{\nameref{subsec:PartialCred}}} as the weighted average of both $D$ and $M$.
\end{rappel_enhanced}


%\columnbreak
\subsection{Full Credibility}\label{subsec:FullCred}
\begin{rappel_enhanced}[Contexte]
The classical credibility approach determines the \textit{\textbf{minimum} data size} required for the experience data ($D$) to be given \textbf{\textit{full credibility}}. The minimum data size, or \textit{\textbf{standard for full credibility}}, depends on the \textbf{loss measure}.
\end{rappel_enhanced}


\subsubsection{Claim Frequency}
The claim frequency random variable $N$ has mean $\mu_{N}$ and variance $\sigma^{2}_{N}$. 

If we assume $N \approx \mathcal{N}(\mu_{N}, \sigma^{2}_{N})$, then the probability of observing claim frequency \textbf{within $k$ of the mean} is \lfbox[formula]{$\Pr(\mu_{N} - k\mu_{N} \leq N \leq \mu_{N} + k\mu_{N}) = 2 \Phi\left(\frac{k \mu_{N}}{\sigma_{N}}\right) - 1$}.

\bigskip

We often assume that the claim frequency \lfbox[conditions]{$N \sim \text{Pois}(\lambda_{N})$} and then apply the normal approximation to find the standard for full credibility for claim frequency \lfbox[formula]{$\lambda_{F}$}. First, we impose that the probability of the claim being with $k$ of the mean must be at least $1 - \alpha$. Then, we rewrite \lfbox[conditions]{$\frac{k \mu_{N}}{\sigma_{N}} = k\sqrt{\lambda_{N}}$} and set \lfbox[formula]{$\lambda_{N} \geq \left(\frac{z_{1 - \alpha/2}}{k}\right)^{2}$} where \lfbox[conditions]{$\lambda_{F} = \left(\frac{z_{1 - \alpha/2}}{k}\right)^{2}$}.


\subsubsection{Claim Severity}
We assume that the loss amounts $X_{1}, X_{2}, \dots, X_{N}$ are independent and identically distributed random variables with mean $\mu_{X}$ and variance $\sigma^{2}_{X}$. Full credibility is attributed to \lfbox[conditions]{$D = \bar{X}$} if \lfbox[conditions]{$2\Phi\left(\frac{k \mu_{X}}{\sigma_{N}/\sqrt{N}}\right) - 1 \geq 1 - \alpha$}. 

\bigskip

Similarly to claim frequency, we apply the normal approximation with \lfbox[conditions]{$\bar{X} \approx \mathcal{N}\left(\mu_{X}, \sigma^{2}_{X}/N\right)$}. Then, we find \lfbox[formula]{$N \geq \left(\frac{z_{1 - \alpha/2}}{k}\right)^{2} \cdot \left(\frac{\sigma_{X}}{\mu_{X}}\right)^{2} = \lambda_{F} CV_{X}^{2}$} where the \textbf{\textit{standard for full credibility for claim severity}} is $\lambda_{F}CV_{X}^{2}$.


\subsubsection{Aggregate Loss}
For the aggregate loss $S = X_{1} + X_{2} + \hdots + X_{N}$, we have \lfbox[formula]{$\mu_{S} = \mu_{N} \mu_{X}$} and \lfbox[formula]{$\sigma^{2}_{S} = \mu_{N} \sigma^{2}_{X} + \mu_{X}^{2} \sigma^{2}_{N}$}.

\bigskip

With the same normality assumptions for the Poisson distributed $N$, we find \lfbox[formula]{$\lambda_{N} \geq \left(\frac{z_{1 - \alpha/2}}{k}\right)^{2} \cdot \left(\frac{\mu_{X}^{2} + \sigma^{2}_{X}}{\mu_{X}^{2}}\right) = \lambda_{F} (1 + CV_{X}^{2})$} where the \textbf{\textit{standard for full credibility for claim severity}} is $\lambda_{F}(1 + CV_{X}^{2})$.

\paragraph{Note}	The conditions are the same for the \textit{\textbf{Pure Premium}} as for the aggregate loss.


%\columnbreak
\subsection{Partial Credibility}\label{subsec:PartialCred}
The \textbf{\textit{credibility factor}} for :
\begin{description}
	\item[Claim Frequency]	is \lfbox[formula]{$Z = \sqrt{\frac{\lambda_{N}}{\lambda_{F}}}$}.
	\item[Claim Severity]	is \lfbox[formula]{$Z = \sqrt{\frac{N}{\lambda_{F} CV_{X}^{2}}}$}.
	\item[Aggregate Loss and Pure Premium]	is \lfbox[formula]{$Z = \sqrt{\frac{\lambda_{N}}{\lambda_{F}(1 + CV_{X}^{2})}}$}
\end{description}


\newpage
\section{Bühlmann Credibility}\label{sec:Buhl}
\subsection{Basic framework}


\subsection{Variance components}



\subsection{Credibility factors}
%Bûhlmann and Bûhlmann-Straub factors
%Estimate loss



\newpage
\section{Bayesian Credibility}\label{sec:Bayes}
\subsection{Basic framework}


\subsection{Premium}



\subsection{Conjugate distributions}

\subsection{Nonparametric empirical Bayes method}




\newpage
\part{Linear Mixed Models}\label{part:LMM}




\newpage
\part{Bayesian Analysis and Markov Chain Monte Carlo}\label{part:BAandMCMC}




\newpage
\part{Statistical Learning}\label{part:statLearn}
\section{K-Nearest Neighbors}\label{sec:KNN}


\newpage
\section{Decision Trees}
%	building, purpose, pruning
%	bagging, random forests, boosting


\newpage
\section{Principal Components Analysis (PCA)}\label{sec:PCA}
%	purpose and computations
%	interpretation of software outputs	


\newpage
\section{Clustering}
%	purpose and computations
%	interpretation of software outputs





\end{multicols*}
\end{document}
