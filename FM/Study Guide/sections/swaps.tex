\chapter[Topic: Interest Rate Swaps]{Topic: Interest Rate Swaps (0-10\%)}

\subsection{Information}

\begin{distributions}[Objective]
The Candidate will understand key concepts concerning interest rate swaps, and how to perform related calculations.
\end{distributions}

\begin{outcomes}[Learning outcomes]
The candidate will be able to:
\begin{enumerate}[label = \alph*)]
	\item	Define and recognize the \textit{definitions} of the following terms:
		\begin{itemize}[leftmargin = *]
		\item	Swap rate;
		\item	Swap term (tenor);
		\item	Notional amount;
		\item	Market value of a swap;
		\item	Settlement dates;
		\item	Settlement period;
		\item	Counterparties;
		\item	Deferred swap;
		\item	Amortizing swap;
		\item	Accreting swap;
		\item	Interest rate swap net payments.
		\end{itemize}
	\item	Given sufficient information, calculate: 
		\begin{itemize}[leftmargin = *]
		\item	The market value;
		\item	Notional amount;
		\item	Spot rates or swap rate,
		\end{itemize}
		of an interest rate swap
		\begin{itemize}[leftmargin = *]
		\item	deferred or otherwise;
		\item	with either constant or varying notional amount.
		\end{itemize}
\end{enumerate}
\end{outcomes}

\begin{ASM_chapter}[Related lessons ASM]
Section 11: Interest Rate Swaps
\begin{itemize}[leftmargin = *]
	\item	\nameref{L.-11b}
\end{itemize}
\end{ASM_chapter}

\subsection{Chapter summaries}

\begin{CHPT_SUMM_AUTO}[label = {L.-11b}]{11b. What is an Interest Rate Swap?}
Business loans are often made on a variable or adjustable rate basis.
Thus, the rate "floats" from period to period according to some kind of index.

One such index is the \textbf{prime interest rate} which is the rate at which US banks will lend money to their best customers.

If a company is concerned that the index could increase to the point where it would have difficulties making payments. 
If the company wants to be certain of the amounts of its future payments, it would rather a fixed interest rate. 

It could enter an \textbf{interest rate swap} which permits the company to "swap", or exchange, the variable rate in its loan agreement for a known, fixed rate. 
Thus the company (A) arranges the swap with a \textbf{third party} (not the lender!) company B. Company A continues paying the lender the variable amount of interest each period.

Companies A and B are the \textbf{counterparties} to the interest rate swap. 
Under the terms of the swap, Company A will pay Company B a fixed amount of interest every period, and Company B will pay Company the variable amount of interest every period.

In actual practice, rather than having both companies make payments every period, one of them will pay the \textbf{net amount} to the other---the \textbf{net swap payment}.

Thus, after all the payments between the parties are netted:
\begin{itemize}[leftmargin = *]
	\item	The original lender receives the variable amount of interest every period.
	\item	Company A ends up paying the fixed amount of interest every period.
	\item	Company B ends up with either:
		\begin{itemize}[leftmargin = *]
		\item	A \underline{profit} equal to the excess of the fixed amount of interest over the variable amount of interest; or
		\item	a \underline{loss} equal to the excess of the variable amount of interest over the fixed amount of interest.
		\end{itemize}
\end{itemize}



\end{CHPT_SUMM_AUTO}