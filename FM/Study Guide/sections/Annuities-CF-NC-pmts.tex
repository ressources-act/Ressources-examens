\chapter[Topic: Annuities / cash flows with non-contingent payments]{Topic: Annuities / cash flows with non-contingent payments (exam weight)}

\subsection{Information}

\begin{distributions}[Objective]
The Candidate will be able to calculate present value, current value, and accumulated value for sequences of non-contingent payments.
\end{distributions}

\begin{outcomes}[Learning outcomes]
The candidate will be able to:
\begin{enumerate}[label = \alph*)]
	\item	Define and recognize the \textit{definitions} of the following terms:
		\begin{multicols*}{2}
		\begin{itemize}[leftmargin = *]
		\item	Annuity-immediate;
		\item	Annuity-due;
		\item	Perpetuity;
		\item	Payable $m$-thly or continously;
		\item	Level payment annuity;
		\item	Arithmetic increasing/decreasing annuity;
		\item	Geometric increasing/decreasing annuity;
		\item	Term of annuity;
		\end{itemize}
		\end{multicols*}
	\item	For each of the following types of annuity / cash flows, given sufficient information of :
		\begin{multicols*}{2}
		\begin{itemize}[leftmargin = *]
		\item	Immediate or due;
		\item	Present value;
		\item	Futur value;
		\item	Current value;
		\item	Interest rate;
		\item	Payment amount;
		\item	Term of annuity,
		\end{itemize}
		\end{multicols*}
		calculate any remaining item. 
	\item[]	The types are:
		\begin{itemize}[leftmargin = *]
		\item	Level annuity, finite term;
		\item	Level perpetuity;
		\item	Non-level annuities / cash flows;
			\begin{itemize}
			\item	Arithmetic progression, finite term and perpetuity;
			\item	Geometric progression, finite term and perpetuity;
			\item	Other non-level annuities / cash flows.
			\end{itemize}
		\end{itemize}
\end{enumerate}
\end{outcomes}

\begin{ASM_chapter}[Related lessons ASM]
Section 3: Annuities
\begin{itemize}
	\item	\nameref{L.-3a}
	\item	\nameref{L.-3b}
	\item	\nameref{L.-3c}
	\item	\nameref{L.-3d}
	\item	\nameref{L.-3e}
	\item	\nameref{L.-3f}
	\item	\nameref{L.-3g}
	\item	\nameref{L.-3h}
	\item	\nameref{L.-3i}
\end{itemize}
Section 4: Complex Annuities
\begin{itemize}
	\item	\nameref{L.-4a}
	\item	\nameref{L.-4b}
	\item	\nameref{L.-4c}
	\item	\nameref{L.-4d}
	\item	\nameref{L.-4e}
	\item	\nameref{L.-4f}
	\item	\nameref{L.-4g}
	\item	\nameref{L.-4h}
	\item	\nameref{L.-4i}
	\item	\nameref{L.-4j}
	\item	\nameref{L.-4k}
	\item	\nameref{L.-4l}
	\item	\nameref{L.-4m}
\end{itemize}
\end{ASM_chapter}
%
%\begin{YTB_vids}[Vidéos YouTube]
%\begin{itemize}
%	\item	
%\end{itemize}
%\end{YTB_vids}

\subsection{Résumés des chapitres}

\begin{CHPT_SUMM_AUTO}[label = {L.-3a}]{3a. The Geometric Series Trap}
Remember the formula for geometric series in words:
\begin{align*}
	r^{10} + r^{20} + \dots + r^{10n}
	&=	r^{10}\frac{1 - r^{n}}{1 - r}
	&=	(\text{first term}) \frac{1 - (\text{ratio})^{\text{nb. of terms}}}{1 - (\text{ratio})}
\end{align*}
	\begin{itemize}
		\item	
	\end{itemize}
\end{CHPT_SUMM_AUTO}

\begin{CHPT_SUMM_AUTO}[label = {L.-3b}]{3b. Annuity-Immediate and Annuity-Due}
	\begin{itemize}
		\item	
	\end{itemize}
\end{CHPT_SUMM_AUTO}

\begin{CHPT_SUMM_AUTO}[label = {L.-3c}]{3c. The Great Confusion: Annuity-Immediate and Annuity-Due}
	\begin{itemize}
		\item	
	\end{itemize}
\end{CHPT_SUMM_AUTO}

\begin{CHPT_SUMM_AUTO}[label = {L.-3d}]{3d. Deferred Annuities}
	\begin{itemize}
		\item	
	\end{itemize}
\end{CHPT_SUMM_AUTO}

\begin{CHPT_SUMM_AUTO}[label = {L.-3e}]{3e. A Short-Cut Method for Annuities with "Block" Payments}
	\begin{itemize}
		\item	
	\end{itemize}
\end{CHPT_SUMM_AUTO}

\begin{CHPT_SUMM_AUTO}[label = {L.-3f}]{3f. Perpetuities}
	\begin{itemize}
		\item	
	\end{itemize}
\end{CHPT_SUMM_AUTO}

\begin{CHPT_SUMM_AUTO}[label = {L.-3g}]{3g. The $\ax{\angl{2n}} / \ax{\angl{n}}$ Trick (and Variations)}
	\begin{itemize}
		\item	
	\end{itemize}
\end{CHPT_SUMM_AUTO}

\begin{CHPT_SUMM_AUTO}[label = {L.-3h}]{3h. What If the Rate Is Unknown?}
	\begin{itemize}
		\item	
	\end{itemize}
\end{CHPT_SUMM_AUTO}

\begin{CHPT_SUMM_AUTO}[label = {L.-3i}]{3i. What If the Rate Varies?}
	\begin{itemize}
		\item	
	\end{itemize}
\end{CHPT_SUMM_AUTO}

\begin{CHPT_SUMM_AUTO}[label = {L.-4a}]{4a. Annuities with "Off-Payments" Part I}
	\begin{itemize}
		\item	
	\end{itemize}
\end{CHPT_SUMM_AUTO}

\begin{CHPT_SUMM_AUTO}[label = {L.-4b}]{4b. Annuities with "Off-Payments" Part II}
	\begin{itemize}
		\item	
	\end{itemize}
\end{CHPT_SUMM_AUTO}

\begin{CHPT_SUMM_AUTO}[label = {L.-4c}]{4c. Avoiding the $m^{\text{thly}}$ Annuity Trap}
	\begin{itemize}
		\item	
	\end{itemize}
\end{CHPT_SUMM_AUTO}

\begin{CHPT_SUMM_AUTO}[label = {L.-4d}]{4d. Continuous Annuities}
	\begin{itemize}
		\item	
	\end{itemize}
\end{CHPT_SUMM_AUTO}

\begin{CHPT_SUMM_AUTO}[label = {L.-4e}]{4e."Double-Dots Cancel" (and so do "upper $m$'s")}
	\begin{itemize}
		\item	
	\end{itemize}
\end{CHPT_SUMM_AUTO}

\begin{CHPT_SUMM_AUTO}[label = {L.-4f}]{4f. A Short Note on Remembering Annuity Formulas}
	\begin{itemize}
		\item	
	\end{itemize}
\end{CHPT_SUMM_AUTO}

\begin{CHPT_SUMM_AUTO}[label = {L.-4g}]{4g. The $\sx{\angln}$ Trap When Interest Variess}
	\begin{itemize}
		\item	
	\end{itemize}
\end{CHPT_SUMM_AUTO}

\begin{CHPT_SUMM_AUTO}[label = {L.-4h}]{4h. Payments in Arithmetic Progression}
	\begin{itemize}
		\item	
	\end{itemize}
\end{CHPT_SUMM_AUTO}

\begin{CHPT_SUMM_AUTO}[label = {L.-4i}]{4i. Remembering Increasing Annuity Formulas}
	\begin{itemize}
		\item	
	\end{itemize}
\end{CHPT_SUMM_AUTO}

\begin{CHPT_SUMM_AUTO}[label = {L.-4j}]{4j. Payments in Geometric Progression}
	\begin{itemize}
		\item	
	\end{itemize}
\end{CHPT_SUMM_AUTO}

\begin{CHPT_SUMM_AUTO}[label = {L.-4k}]{4k. The Amazing Expanding Money Machine (Or Continouss Varying Annuities)}
	\begin{itemize}
		\item	
	\end{itemize}
\end{CHPT_SUMM_AUTO}

\begin{CHPT_SUMM_AUTO}[label = {L.-4l}]{4l. A Short-Cut Method for the Palindromic Annuity}
	\begin{itemize}
		\item	
	\end{itemize}
\end{CHPT_SUMM_AUTO}

\begin{CHPT_SUMM_AUTO}[label = {L.-4m}]{4m. The 0\% Test: A Quick Check of Symbolic Answers}
	\begin{itemize}
		\item	
	\end{itemize}
\end{CHPT_SUMM_AUTO}

%\subsection{Notes sur les vidéos YouTube}

%\begin{YTB_SUMM}[label = {}]{}
%\begin{itemize}
%	\item	
%\end{itemize}
%\end{YTB_SUMM}