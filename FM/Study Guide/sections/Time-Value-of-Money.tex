\chapter[Time Value of Money]{Time Value of Money (10\%-15\%)}

\subsection{Information}

\begin{distributions}[Objective]
The Candidate will understand and be able to perform calculations relating to present value, current value, and accumulated value.
\end{distributions}

\begin{outcomes}[Learning outcomes]
The candidate will be able to:
\begin{enumerate}[label = \alph*)]
	\item	Define and recognize the \textit{definitions} of the following terms:
		\begin{multicols*}{2}
		\begin{itemize}[leftmargin = *]
		\item	Interest rate (rate of interest);
		\item	Simple interest;
		\item	Compound interest;
		\item	Accumulation function;
		\item	Future value;
		\item	Current value;
		\item	Present value;
		\item	Net present value;
		\item	Discount factor;
		\item	Discount rate (rate of discount);
		\item	Convertible $m$-thly (\dots ?);
		\item	Nominal rate;
		\item	Effective rate;
		\item	Inflation;
		\item	Real rate of interest;
		\item	Force of interest;
		\item	Equation of value.
		\end{itemize}
		\end{multicols*}
	\item	Given any 3 of:
		\begin{multicols*}{3}
		\begin{itemize}[leftmargin = *]
		\item	Interest rate;
		\item	Period of time;
		\item	Present value;
		\item	Current value;
		\item	Future value,
		\end{itemize}
		\end{multicols*}
		calculate the remaining item using \textit{simple} or \textit{compound} interest;
	\item[]	Solve time value of money equations involving variable force of interest;
	\item	Given any 1 of:
		\begin{itemize}[leftmargin = *]
		\item	Effective interest rate;
		\item	Nominal interest rate convertible $m$-thly;
		\item	Force of interst,
		\end{itemize}
		calculate any of the other items;
	\item	Write the equation of value given a set of cash flows and interest rate.
\end{enumerate}
\end{outcomes}

\begin{ASM_chapter}[Related lessons ASM]
Section 1: Interest rates and Discount Rates
\begin{itemize}
	\item	\nameref{L.-1a}
	\item	\nameref{L.-1b}
	\item	\nameref{L.-1c}
	\item	\nameref{L.-1d}
	\item	\nameref{L.-1e}
	\item	\nameref{L.-1f}
	\item	\nameref{L.-1g}
	\item	\nameref{L.-1h}
\end{itemize}
Section 2: Practical Applications
\begin{itemize}
	\item	\nameref{L.-2a}
	\item	\nameref{L.-2b}
\end{itemize}
\end{ASM_chapter}

\begin{YTB_vids}[Vidéos YouTube]
\begin{itemize}
	\item	
\end{itemize}
\end{YTB_vids}

\subsection{Résumés des chapitres}

\begin{CHPT_SUMM_AUTO}[label = {L.-1a}]{1a. Basic Concepts}
	\begin{itemize}
		\item	
	\end{itemize}
\end{CHPT_SUMM_AUTO}

\begin{CHPT_SUMM_AUTO}[label = {L.-1b}]{1b. Why Do We Need a Force of Interest?}
	\begin{itemize}
		\item	
	\end{itemize}
\end{CHPT_SUMM_AUTO}

\begin{CHPT_SUMM_AUTO}[label = {L.-1c}]{1c. Defining the Force of Interest}
	\begin{itemize}
		\item	
	\end{itemize}
\end{CHPT_SUMM_AUTO}

\begin{CHPT_SUMM_AUTO}[label = {L.-1d}]{1d. Finding the Fund in Terms of the Force of Interest}
	\begin{itemize}
		\item	
	\end{itemize}
\end{CHPT_SUMM_AUTO}

\begin{CHPT_SUMM_AUTO}[label = {L.-1e}]{1e. The Simplest Case: A Constant Force of Interest}
	\begin{itemize}
		\item	
	\end{itemize}
\end{CHPT_SUMM_AUTO}

\begin{CHPT_SUMM_AUTO}[label = {L.-1f}]{1f. Power Series}
	\begin{itemize}
		\item	
	\end{itemize}
\end{CHPT_SUMM_AUTO}

\begin{CHPT_SUMM_AUTO}[label = {L.-1g}]{1g. The Variable Force of Interest Trap}
	\begin{itemize}
		\item	
	\end{itemize}
\end{CHPT_SUMM_AUTO}

\begin{CHPT_SUMM_AUTO}[label = {L.-1h}]{1h. Equivalent Rates}
	\begin{itemize}
		\item	
	\end{itemize}
\end{CHPT_SUMM_AUTO}

\begin{CHPT_SUMM_AUTO}[label = {L.-2a}]{{2a. Equations of Value, Time Value of Money, and Time Diagrams}}
	\begin{itemize}
		\item	
	\end{itemize}
\end{CHPT_SUMM_AUTO}

\begin{CHPT_SUMM_AUTO}[label = {L.-2b}]{2b. Unknown Time and Unknown Interest Rate}
	\begin{itemize}
		\item	
	\end{itemize}
\end{CHPT_SUMM_AUTO}

\subsection{Notes sur les vidéos YouTube}

%\begin{YTB_SUMM}[label = {}]{}
%\begin{itemize}
%	\item	
%\end{itemize}
%\end{YTB_SUMM}