\documentclass[10pt, french]{article}

%% -----------------------------
%% Préambule
%% -----------------------------
\input{cheatsht-preamble-general.tex}
%% -----------------------------
%% Variable definition
%% -----------------------------
\def\cours{FM}
\def\sigle{(SOA)}
%
% 	Save more space than default
%
\setlength{\abovedisplayskip}{-15pt}
\setlist{leftmargin=*}
\setcounter{secnumdepth}{0}
%
%	Extra math symbols
%
\usepackage{mathrsfs}
\usepackage{multirow}
%
% 	thin space, limits underneath in displays
%
\DeclareMathOperator*{\argmax}{arg\,max} 
\usepackage{stackengine}
\newcommand\cumlaut[2][black]{\stackon[.33ex]{#2}{\textcolor{#1}{\kern-.04ex.\kern-.2ex.}}}

%% -----------------------------
%% 	Colour setup for sections
%% -----------------------------
\def\SectionColor{cobalt}
\def\SubSectionColor{azure(colorwheel)}
\def\SubSubSectionColor{azure(colorwheel)}
%% -----------------------------

\begin{document}

%\begin{center}
%	\textsc{\Large Contributeurs}\\[0.5cm] 
%\end{center}
%\input{contributeurs/contrib-FM}

%\newpage
\raggedcolumns
\begin{multicols*}{2}

Notes calculatrice
\begin{itemize}
	\item	The option \calc{P/Y} permits to set the payments per year and \calc{C/Y} the number of time the interest rate is \textbf{c}ompounded per year.
	\item	In this case, we set the rate to $i^{(m)}$, the number of periods to $n \times m$, the payment to monthly payment $\frac{P}{m}$.
	\item	If we do \calc{2ND} \calc{xP/Y} it multiplies the period by the number of payments per year.
	\item	To calculate decreasing / increasing annuities just calculate each component one at a time.
	\item	To calculate bonds, enter the coupon for the year (for example, if semi-annual sum both coupons).
\end{itemize}


\def\SectionColor{red!80!white}
\section{Time Value of Money}
\begin{distributions}[Notation effective rate of interest]
\begin{description}
	\item[$a(t)$]	\textbf{Accumulation function} defined as the \textit{A}ccumulated \textit{V}alue (\textit{AV}) of the fund at time $t$ of an initial investment of $\$1$ at time 0.
		\begin{itemize}
		\item	$a(0) \equiv 1$.
		\item	Generally \textbf{continuous} and \textbf{increasing}.
		\end{itemize}
	\item[$A(t)$]	\textbf{Amount function} defined as the Accumulated Value (AV) of the fund at time $t$ of an initial investment of $\$k$ at time 0.
		\begin{itemize}[leftmargin = *]
		\item	$A(t) = k a(t)$.
		\end{itemize}
	\item[$i_{t}$]	\textbf{Effective rate of interest} defined as the rate of growth based on the amount in the fund at the \textbf{\textit{beginning}} of the year.
		\begin{itemize}[leftmargin = *]
		\item	$i_{t} = \frac{A(t) - A(t - 1)}{A(t - 1)}$.
		\item	We deduce $A(t) = (1 + i_{t})A(t - 1)$.
		\end{itemize}
\end{description}
\end{distributions}

We then find: 
\begin{description}
	\item[$a(t) - a(t - 1)$]	\textbf{\textit{Amount} of growth} in the $t^{\text{th}}$ year.
		\begin{itemize}[leftmargin = *]
		\item	a.k.a. the interest earned
		\end{itemize}
	\item[$\frac{a(t) - a(t - 1)}{a(t - 1)}$]	\textbf{\textit{Rate} of growth} in the $t^{\text{th}}$ year.
		\begin{itemize}[leftmargin = *]
		\item	a.k.a. effective rate of interest denoted $i_{t}$.
		\end{itemize}
\end{description}

\

Discounting is finding the price we'd be willing to pay for the promise to receive a future amount. That is to say, finding the present value which is why $i = \frac{d}{1 - d}$.
\begin{distributions}[Notation effective rate of discount]
\begin{description}
	\item[$d_{t}$]	\textbf{Effective rate of discount} defined as the rate of growth based on the amount in the fund at the \textbf{\textit{end}} of the year.
		\begin{itemize}
		\item	$d_{t} = \frac{A(t) - A(t - 1)}{A(t)}$.\
		\item	Although we could get by without it, it's useful to determine the amount to pay today for a specified amount in the future.
		\end{itemize}
\end{description}
\end{distributions}

We then find:
\begin{align*}
	v 
	&=	(1 - d) 
	=	\frac{1}{1 + i}	\\
	d 
	&=	\frac{i}{1 + i}
\end{align*}

\begin{distributions}[Notation nominal rates of interest]
\begin{description}
	\item[$i^{(m)}$]	\textbf{Nominal} annual rate of of interest \textbf{compounded $m$ times a year}.
	\item[$\frac{i^{(m)}}{m}$]	\textbf{Effective} rate of of interest \textbf{for an $m^{\text{th}}$ of a year}.
		\begin{itemize}[leftmargin = *]
		\item	Thus $(1 + i) = \left(1 + \frac{i^{(m)}}{m}\right)^{m}$.\
		\end{itemize}
\end{description}
\end{distributions}

\

The force of interest is the rate of growth at a specific point in time.
\begin{distributions}[Notation force of interest]
\begin{description}
	\item[$\delta_{t}$]	The \textbf{Force of interest} at time $t$.
		\begin{itemize}[leftmargin = *]
		\item	$\delta_{t} = \frac{A'(t)}{A(t)}$.
		\end{itemize}
	\item[$\delta$]	The \textbf{constant} force of interest.
		\begin{itemize}[leftmargin = *]
		\item	a.k.a. the nominal rate of interest compounded continuously.
		\item	$\delta = \underset{m \rightarrow \infty}{\lim} i^{(m)} = i^{(\infty)} = \ln(1 + i)$.
		\end{itemize}
\end{description}
\end{distributions}

We then find:
\begin{align*}
	a(t)
	&=	\textrm{e}^{\int_{0}^{t}\delta_{r}dr}	\\
	FV
	&=	\textrm{e}^{\int_{t_{1}}^{t_{2}}\delta_{r} dr}	\\
	&\equiv	\frac{a(t_{2})}{a(t_{1})}	
\end{align*}

and for a constant force of interest: 
\begin{align*}
	a(t)	
	&=	\textrm{e}^{\int_{0}^{t}\delta dr}	
	=	\textrm{e}^{\delta t}
\end{align*}


\newpage
\def\SectionColor{blue!80!white}
\section{Annuities / cash flows with non-contingent payments}
\textbf{Geometric series relation :}
\begin{align*}
	r^{10} + r^{20} + \dots + r^{10n}
	&=	r^{10}\frac{1 - r^{n}}{1 - r}	\\
	&=	(\text{first term}) \frac{1 - (\text{ratio})^{\text{nb. of terms}}}{1 - (\text{ratio})} 	\\
\end{align*}

\begin{definitionNOHFILL}[Annuity]
An annuity is called an \textbf{annuity-immediate} if, in determining its present value, the valuation date is \textbf{\textit{one period before}} the first payment (symbol $\ax{\angln}$). An annuity is called an \textbf{annuity-due} if, in determining its present value, the valuation date is \textit{\textbf{on}} the date of the first payment (symbol $\ax*{\angln}$).

\tcbline

An annuity is called an \textbf{annuity-immediate} if, in determining its accumulated value, the valuation date is \textit{\textbf{on}} the date of the last payment (symbol $\sx{\angln}$). An annuity is called an \textbf{annuity-due} if, in determining its accumulated value, the valuation date is \textit{\textbf{one period after}} the date of the last payment (symbol $\sx*{\angln}$).
\end{definitionNOHFILL}

\textbf{Standard Annuities:}
\begin{align*}
	\cumlaut[cyan]{a}_{\angl{n}}^{\textcolor{red}{(m)}} 
		&= \frac{1 - v^n}{ (i^{\textcolor{red}{(m)}}{\color{black}|}{\color{cyan}d}^{\textcolor{red}{(m)}}{\color{black})}}	&
	\cumlaut[cyan]{s}_{\angln}^{\textcolor{red}{(m)}} 
		&=	\frac{(1+i)^{n} - 1}{(i^{\textcolor{red}{(m)}} | \textcolor{cyan}{d}^{\textcolor{red}{(m)}})}	\\
\end{align*}

\textbf{Annuities for Payments in Arithmetic Progression:}
\begin{align*}
	(I^{\textcolor{blue}{(m)}}\cumlaut[cyan]{a})_{\angl{n}}^{\textcolor{red}{(m)}} 
		&= \frac{\cumlaut[black]{a}_{\angl{n}}^{\textcolor{blue}{(m)}} - nv^n}{(i | \textcolor{cyan}{d}^{\textcolor{red}{(m)}})} &
	(D^{\textcolor{blue}{(m)}}\cumlaut[cyan]{a})_{\angl{n}}^{\textcolor{red}{(m)}} 
		&= \frac{n - a_{\angl{n}}^{\textcolor{blue}{(m)}}}{(i | \textcolor{cyan}{d}^{\textcolor{red}{(m)}})}
\end{align*}

\begin{align*}
	(I^{\textcolor{blue}{(m)}}\cumlaut[cyan]{s})_{\angl{n}}^{\textcolor{red}{(m)}} 
		&= \frac{\cumlaut[black]{s}_{\angl{n}}^{\textcolor{blue}{(m)}} - n}{(i | \textcolor{cyan}{d}^{\textcolor{red}{(m)}})} &
	(D^{\textcolor{blue}{(m)}}\cumlaut[cyan]{s})_{\angl{n}}^{\textcolor{red}{(m)}} 
		&= \frac{n(1+i)^{n} - s_{\angl{n}}^{\textcolor{blue}{(m)}}}{(i | \textcolor{cyan}{d}^{\textcolor{red}{(m)}})}  	\\
\end{align*}

\textbf{Annuities for Payments in Arithmetic Progression:}
\begin{align*}
	(\bar{I}\bar{s})_{\angl{n}i} &= \frac{\bar{s}_{\angl{n}i} - n}{\delta} &
	(\bar{D}\bar{s})_{\angl{n}i} &= \frac{n e^{\delta n} - \bar{s}_{\angl{n}i}}{\delta} \\
	(\bar{I}\bar{a})_{\angl{n}i} &= \frac{\bar{a}_{\angl{n}i} - n e^{-\delta n}}{\delta} &
	(\bar{D}\bar{a})_{\angl{n}i} &= \frac{n - \bar{a}_{\angl{n}i}}{\delta} 	\\
\end{align*}	

\textbf{Annuities for Payments in Arithmetic Progression:}
\begin{align*}
	\cumlaut[cyan]{a}_{\angl{n}i^{R}}
		&=	\frac{1 - (1 + i^{R})^{-n}}{\left( \frac{i^{R}}{1 + i^{R}} \right)} {\color{cyan}\frac{1}{1 + r}	}	&
	{s}_{\angln i^{R}}
		&=	\frac{(1+i)^{n} - (1 + r)^{n}}{i - r} 	\\
\end{align*}

\textbf{Relations between types of Annuities:}
\begin{align*}
	\ax**{\angln}	
	&=	(1 + i) \ax{\angln}	\\
	&=	\ax{\angl{n - 1}} - 1	\\
	\sx**{\angln}	
	&=	(1 + i) \sx{\angln}	\\
	&=	\sx{\angl{n + 1}} - 1	
\end{align*}

\textbf{Relations for deferred Annuities:}
\begin{align*}
	\ax[r|]{n} 
	&\equiv \ax**[r + 1| ]{n}	\\
	\ax[r|]{n}
	&=	\ax{n + r} - \ax{r}	\\
	\ax[r|]{n}
	&=	v^{r} \ax{n}		\\
	&\equiv	v^{r + 1} \ax**{n}	\\
\end{align*}

\textbf{Other relations:}
\begin{align*}
	\ax{\angl{2n}} / \ax{\angl{n}}
	&=	(1 + v^{n})	\\
	\ax{\angl{3n}} / \ax{\angl{n}}
	&=	(1 + v^{n} + v^{2n})	\\
\end{align*}

\textbf{Perpetuities:}
\begin{align*}
	\cumlaut[cyan]{a}_{\angl{\infty}} 
		&= \frac{1}{(i|\textcolor{cyan}{d})} 	\\
	(I\cumlaut[cyan]{a})_{\angl{\infty}} 
		&= \frac{1}{(i|\textcolor{cyan}{d})d} 	\\
\end{align*}


\textbf{Payments in arithmetic progression:}\\
For a first payment of $P$ and a common difference of $Q$, 
\begin{align*}
	PV
	&=	P\ax{\angln} + Q \frac{\ax{\angln} - n v^{n}}{i}	\\
\end{align*}

\newpage
\def\SectionColor{cyan!80!white}
\section{Bonds}
\begin{definitionNOHFILL}[Callable Bond]
Bond in which the borrower (issuer) may redeem the bond at a certain \textbf{call price} before it has matured. Generally, the bond cannot be redeemed before a certain \textbf{call date}. 

\begin{itemize}
	\item	The \textbf{call premium} is the difference between the \textit{call price} and the \textit{redemption value} at maturity.
\end{itemize}

\begin{rappel_enhanced}[Context]
\begin{itemize}
	\item	When interest rates drop, the bond issuer can redeem the bonds and reinvest the call price into bonds with a lower yield rate.
	\item	To compensate for the higher risk, callable bonds typically offer a \textit{higher yield rate}.
\end{itemize}
\end{rappel_enhanced}

To price callable bonds, we find the worst possible date of redemption at which we ensure a minimal return :
\begin{itemize}
		\item	For bonds selling at a \textbf{P}remium, the \textbf{E}arliest redemption date is the \textbf{W}orst one (\textbf{PEW}).
		\item	For bonds selling at a discount, the latest redemption date is the worst one.
\end{itemize}
\end{definitionNOHFILL}



\newpage
\def\SectionColor{teal!80!white}
\section{General Cash Flows and Portfolios}
\subsection{Stocks}
\begin{definitionNOHFILLprop}[Price of a share of Stock]
\begin{distributions}[Notation]
\begin{description}
	\item[$D$]	First dividend.
	\item[$k$]	Annual rate of increase, \lfbox[conditions]{$k	<	i$}.
\end{description}
\end{distributions}
\begin{description}
	\item[Level dividends]	\lfbox[formula]{$P	=	\frac{D}{i}$}.
	\item[Increasing dividends]	\lfbox[formula]{$P	=	\frac{D}{i - k}$}.
\end{description}
\end{definitionNOHFILLprop}


\subsection{Duration and Convexity}
\begin{description}
	\item[$P(i)$]	PV of the CFs at interest rate $i$.
		\begin{align*}
		P(i)	
		&=	\underset{t = 0}{\overset{n}{\sum}} (A_{t}v^{t})
		\end{align*}
\end{description}

\begin{definitionNOHFILLsub}[Duration]
\begin{description}
	\item[$D_{\text{mac}}(i)$]	Macaulay duration \textcolor{teal}{(PV sensitivity)} at interest rate $i$ \textcolor{teal}{(at force of interest $\delta$)}.
		\begin{align*}
		D_{\text{mac}}(i)
		&=	\frac{\underset{t = 0}{\overset{n}{\sum}} (t) (A_{t}v^{t})}{P(i)}	\\
		&=	\textcolor{teal}{\frac{-P'(\delta)}{P(\delta)}}
		\end{align*}
	\item[$D_{\text{mod}}(i)$]	Modified duration \textcolor{teal}{(PV sensitivity)} at interest rate $i$.
		\begin{align*}
		D_{\text{mod}}(i)
		&=	\frac{\underset{t = 0}{\overset{n}{\sum}} (t) (A_{t}v^{t + 1})}{P(i)}	\\
		&=	\textcolor{teal}{\frac{-P'(i)}{P(i)}}
		\end{align*}
\end{description}

\tcbline

$D_{\text{mod}}(i)	=	vD_{\text{mac}}(i)$.
\end{definitionNOHFILLsub}

For a portfolio with $n$ components of duration $D_{i}$ and price $P_{i}$, $D(ptf)	=	\frac{\sumz{n}{i = 1}D_{i}P_{i}}{\sumz{n}{i = 1}P_{i}}$.


\begin{definitionNOHFILLprop}[Approximations]
\begin{distributions}[Notation]
\begin{description}
	\item[$i_{0}$]	Original interest rate used to calculate the PV of the cash flows.
	\item[$i_{n}$]	New interest rate for which we want to approximate the change in Pv.
\end{description}
\end{distributions}
\begin{description}
	\item[First-Order Modified Price Approximation]	\lfbox[formula]{$P(i)	\approx	P(i_{0})[1 - (i - i_{0})D_{\text{mod}}(i_{0})]$}.
	\item[First-Order Macaulay Price Approximation]	\lfbox[formula]{$P(i)	\approx	P(i_{0}) \left( \frac{1 + i_{0}}{1 + i} \right)^{D_{\text{mac}}(i_{0})}$}.
\end{description}
\end{definitionNOHFILLprop}

\begin{definitionNOHFILLsub}[Convexity]
\begin{description}
	\item[$C_{\text{mac}}(i)$]	Macaulay convexity at interest rate $i$.
		\begin{align*}
		C_{\text{mac}}(i)
		&=	\frac{\underset{t = 0}{\overset{n}{\sum}} (t^{2}) (A_{t}v^{t})}{P(i)}	\\
		&=	\frac{P''(\delta)}{P(\delta)}	
		\end{align*}
	\item[$C_{\text{mod}}(i)$]	Modified convexity at interest rate $i$.
		\begin{align*}
		C_{\text{mod}}(i)
		&=	\frac{\underset{t = 0}{\overset{n}{\sum}} (t) (t + 1) (A_{t}v^{t + 2})}{P(i)}	\\
		&=	\frac{P''(i)}{P(i)}	
		\end{align*}
\end{description}

\tcbline

$C_{\text{mod}}(i)	=	v^{2}\left(C_{\text{mac}}(i) + D_{\text{mac}}(i)\right)$
\end{definitionNOHFILLsub}


\newpage
\def\SectionColor{gray!80!white}
\section{Immunization}

\begin{center}
\begin{tabular}{| >{\columncolor{white} }m{3cm} | >{\columncolor{white} }m{3cm}  |}
\hline\rowcolor{airforceblue} 
\textcolor{white}{\textbf{Redington}}	&	\textcolor{white}{\textbf{Full}}		\\\specialrule{0.1em}{0em}{0em} 
\multicolumn{2}{c}{$PV_{\text{A}}	=	PV_{\text{L}}$}	\\\hline
\multicolumn{2}{c}{$MacD_{\text{A}}	=	MacD_{\text{L}}$}	\\
\multicolumn{2}{c}{\textit{or}}	\\
\multicolumn{2}{c}{$P'_{\text{A}}	=	P'_{\text{L}}$}	\\\hline
$C_{\text{A}}	=	C_{\text{L}}$	&	There has to be an 	\\
\textit{or}	&	$A$ CF before and \\
$P''_{\text{A}}	>	P''_{\text{L}}$	&	after each $L$ CF	\\\hline
Immunizes against small changes in $i$	&	Immunizes against any change in $i$	\\\hline
\end{tabular}
\end{center}


\end{multicols*}
\end{document}
